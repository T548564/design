% !TeX root = ../main.tex
% -*- coding: utf-8 -*-

\chapter{基于对抗生成网络特征提取的近边界数据研究}\label{3}

\section{近边界对抗性样本}

在\ref{5}\ref{5.3}中,本文通过大量的实验证明了近边界数据在大多数模型窃取攻击中,其近边界特征在盗窃模型中被保留。因此,近边界数据可以作为推断深度神经网络模型所有权的依据使用。下面给出近边界数据的定义:

\begin{myDef}
	\label{def:1}
	近边界数据。给定一个数据样本$x$,一个阈值$\theta$,如果数据样本$x$满足$\vert g_i(x) - g_j(x) \vert \leq \theta$,其中$i \neq j $并且$min(g_i(x), g_j(x)) \geq max_{k \neq i, j}g_k(x)$,$g_k(x)$代表数据样本$x$决策为类别$k$的概率,则数据样本$x$被称为近边界数据。
\end{myDef}


\section{CW生成近边界对抗性样本}

尽管近边界在模型的知识产权保护中表现出显著的效果,但是自然的近边界数据在样本空间中的占比很低,甚至可以忽略,因此如何得到一定规模的近边界数据样本仍然很困难。

根据最近的一些研究\cite{cao2021ipguard},对抗性样本通常被用于确定分类器的分类边界。具体而言,对抗性样本有两个分类:原始分类和目标分类。其中,原始分类是该样本不经过特殊处理的原始分类结果,目标分类是对原始样本添加微小噪声后的分类结果。如图\ref{原始样本与对抗性样本对比}所示,对抗性样本对分类边界的跨越体现在,在视觉上对抗性样本和原始样本几乎没有差别,但是分类结果却是目标分类。

\begin{figure}[htbp]%%图,[htbp]是浮动格式
	\begin{minipage}[t]{0.5\linewidth}        %图片占用一行宽度的50%
		\hspace{2mm}
		\center
		\includegraphics[width=4.5cm,height=3cm]{对抗性样本原图}
		\centerline{原始样本}
	\end{minipage}
	\begin{minipage}[t]{0.5\linewidth}        %图片占用一行宽度的50%
		\hspace{2mm}
		\center
		\includegraphics[width=4.5cm,height=3cm]{对抗性样本}
		\centerline{对抗性样本}
	\end{minipage}
\setlength{\abovecaptionskip}{7mm} %图片标题与图片距离
\caption{原始样本与对抗性样本对比}
\label{原始样本与对抗性样本对比}
\end {figure}

本文认为该特征可以帮助从对抗性样本中获得较多的近边界数据。因此,本文测试了几种常用的生成对抗性样本的方法,以帮助我们构建近边界数据。

\noindent$\bm{Fast \ Gradient \ Sign \ Method(FGSM):}$FGSM \cite{goodfellow2014explaining}是最经典的构建对抗性样本的方法之一,它是一种基于梯度生成对抗性样本的方法,属于无目标攻击方式。只需要对原始样本添加微小的扰动$\eta$,如\ref{eq:3.1},\ref{eq:3.2}所示,即可生成样本$x$的对抗性样本$\tilde{x}$。

\begin{equation}
	\label{eq:3.1}
	\eta = \epsilon \cdot sign(\bigtriangledown_xJ(\theta,x,y))
\end{equation}

\begin{equation}
	\label{eq:3.2}
	\tilde{x} = clip(x + \eta)
\end{equation}

\noindent 其中$sign$是符号函数,$x$表示原始样本,$y$表示$x$的真实类别,$\theta$表示模型权重参数,$J$表示分类器损失函数,$\bigtriangledown_x$表示对原始样本$x$求偏导,$clip$是将样本投射回可行数据域,$\epsilon$用来控制变化幅度。

FGSM 生成对抗性样本的速度非常快,但其结果非常依赖$\epsilon$的选择,因此探索不同的$\epsilon$是使用该方法的重点。除此之外,我们还测试了许多FGSM 的进阶版本如IGSM和RFSGM, 它们引入了迭代加入噪声和弱扰动的方法。IGSM 迭代式地使样本跨越分类边界直至成功,RFGSM 则是增加了扰动的多样性,可以更精细地生成对抗性样本。在实际结果中我们发现FGSM 生成对抗性示例尽管速度非常快,但位于分类边界附近的数据比例却极低。IGSM 和RFGSM 效果要比FGSM 好,但仍认为不符合我们的期望。在大量的测试中,我们发现CW能够生成大量在分类边界附近的样本,具体的测试结果在\ref{5}\ref{5.2}。

\noindent$\bm{Carlini \ and \ Wagner's \ methods(CW):}$CW \cite{carlini2017towards}方法同样是添加噪声到对抗性样本中,但其具有三种变体:$CW$-$L_0$,$CW$-$L_2$和$CW$-$L_{\infty}$,不同的变体使用不同的方法来衡量噪声的大小,其中$CW$-$L_2$在实验中效果最为突出,因此本文使用该方法作为生成对抗性样本的选择。具体而言,$CW$-$L_2$对于给定的初始样本迭代搜索一个小噪声使示例变为对抗性样本,这种思路使得生成的对抗性样本都集中在分类边界附近,但相应地,$CW$-$L_2$牺牲了效率。
\section{近边界数据私有化}


\section{本章小结}

