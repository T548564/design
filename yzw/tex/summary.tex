% !TeX root = ../main.tex
% -*- coding: utf-8 -*-
\chapter{总结与展望}\label{6}


\section{工作总结}

DNN在给人类社会生活带来便利的同时,也带来了严重的知识产权问题,模型水印和模型指纹是当前保护DNN知识产权的两种主要方法。

在本文中,我们讨论了使用模型水印和模型指纹验证模型所有权的局限性,提出了用推断模型所有权代替验证所有权的新思路。我们认为可以从数据驱动的角度抵御模型盗窃,即在源模型上找到一种可以量化的属性,并且这种属性会被源模型派生出的模型继承,那么就可以从这个角度设计算法来推断模型的所有权。

从数据驱动的角度,本文提出了一种基于近边界数据的模型所有权推断方法。该方法首先比较并选择CW-$L_2$方法生成近边界对抗性样本。为了防止他人轻易复制近边界数据,我们考虑将其私有化,所以训练了一种基于DCGAN的近边界生数据成器用以将近边界数据私有化和扩展,实验测试了生成器能够显著地学习近边界数据的特征并生成新的数据。为了提升推断所有权的置信度,我们设计了新的损失函数微调源模型分类边界,得到最终版本的源模型和近边界数据。最后提出使用假设检验的方法来比对私有近边界数据和其他近边界数据的结果,成功推断模型所有权。

我们在CIFAR-10,Heritage和Intel\_image这三个公开数据集上针对生成初始近边界数据的方法选择,数据近边界特性评估,微调目标分类边界的影响,推断模型所有权的有效性和不同规模近边界数据的可伸缩性扩展这几个方面做了详细的测评和分析,实验证明了我们的方法可以高置信度地推断模型所有权,同时对不同的模型盗窃方法具有很强的鲁棒性。



\section{工作展望}

如何合理有效的保护模型的知识产权已经成为DNN领域的热点研究方向,本文提出数据驱动来推断模型所有权代替一般的验证所有权。本文提出了基于近边界数据的模型所有权推断方法,并展现出了不错的效果,但仍存在一些不足:

\begin{enumerate}
	\renewcommand{\labelenumi}{\theenumi)}
	\item 本文提出的方法主要针对小分类情况下的DNN分类模型。在大分类的情况下,如何选择合适的分类边界计算距离值得探讨。如果大分类模型被迁移到小分类模型上引起类别发生变化,原始的类别应该如何映射到新类别。因此未来的工作应该加入大分类情况下研究。
	\item 本文提出的方法主要是针对DNN分类模型的。对于其他的DNN模型,如何找到类似分类模型分类边界概念来做具体的量化计算是数据驱动推断模型所有权的关键所在。
	\item 虽然本文对CW-$L_2$方法进行了改进,一定程度上加快了算法的效率,但是算法整体由于二分查找加迭代的方式仍然显得效率低下。在未来的工作中,应该探索出一种效果相当但是效率更快的方式生成近边界数据。
\end{enumerate}

综上,本文提出的方法还有很大的探索空间。除此之外,未来的工作应该研究的更多不限于模型水印和指纹的方法,以更好的保护DNN模型的知识产权。





