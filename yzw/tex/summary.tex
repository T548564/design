% !TeX root = ../main.tex
% -*- coding: utf-8 -*-
\chapter{总结与展望}\label{6}

本章对本文提出的基于近边界数据的模型所有权推断方法进行总结,主要包括研究背景、存在问题,实现方案以及主要贡献。然后通过分析方法的不足之处,提出对未来工作的展望。

\section{工作总结}

随着科技的不断发展,深度神经网络模型逐渐成熟,并在社会发展中扮演着日益重要的角色。然而,由于训练成熟、高性能的DNN模型需要昂贵的成本,不法分子开始对这些模型发起窃取攻击,带来了严重的知识产权问题。

本文主要针对神经神经网络模型的知识产权保护方法进行研究。模型水印和模型指纹是目前解决模型知识产权问题的两种主要方法,通过相关工作的调研发现,这两种方法在验证模型所有权时很难抵御歧义攻击。针对上述问题,本文提出了使用数据驱动推断模型所有权,代替传统验证所有权的新思路。本文认为可以从数据驱动的角度抵御模型盗窃,即在源模型上找到一种可以量化的属性,如果这种属性会被源模型派生出的模型所继承,那么就可以从这个角度设计算法来推断模型的所有权。根据这个思想,本文构造了一类特殊的数据——近边界数据,作为推断所有权的依据。本文的主要贡献如下:

1)提出基于数据推断所有权代替验证所有权,解决验证所有权带来的歧义攻击问题。推断模型所有权是比较某类数据在模型上的最优性,最优者推断获得该模型的所有权。这种方式并不是去验证特定的水印或指纹,结果的可比性和唯一性可以有效避免歧义攻击。。

2)提出近边界数据这一特殊数据,作为推断模型所有权的依据。本文基于三个公开数据集在盗窃模型和无关模型上做了充分的测试,验证了近边界数据的近边界性可以被盗窃模型所继承,而在无关模型上不会有近边界的特点,因此可以作为推断所有权的依据。利用对抗性样本靠近模型分类边界的特点,在CW-$L_2$算法的基础上,实现本文生成初始近边界数据的算法。实验表明此算法生成的数据足够靠近分类边界,满足推断所有权的要求。

3)对近边界数据进行私有化处理并基于处理后的数据对源模型进行微调,针对各种模型盗窃技术增强本文方法的性能和防御性。为了防止近边界数据被轻易伪造,设计了基于DCGAN的特征提取器,提取近边界数据特征后,使用其生成器生成新的、私有化的近边界数据。在此基础之上,重新设计新的损失函数微调源模型,使近边界数据更加靠近目标分类边界,成功推断模型所有权的置信度达95\%以上。在三个公开数据集和主流盗窃模型上的实验证明了本文提出的方法在推断模型所有权时的有效性和鲁棒性。


\section{未来展望}

如何合理有效的保护模型的知识产权已经成为DNN领域的热点研究方向,本文提出数据驱动推断模型所有权为模型知识产权保护提供了新思路。但是,本文仍存在一些不足之处:

1)虽然本文对CW-$L_2$方法进行了改进,一定程度上加快了算法的效率,但是算法整体由于二分查找加迭代的方式仍然显得效率低下。在未来的工作中,应该探索出一种效果相当但是效率更快的方法生成近边界数据。

2)本文提出的方法主要针对小分类情况下的DNN分类模型。在大分类的情况下,如何选择合适的分类边界计算数据到分类边界的距离值得探讨。如果大分类模型被迁移到小分类模型上引起类别发生变化,原始的分类边界应该如何映射到新的分类边界。因此未来的工作应该加入对大分类情模型的研究。

3)本文提出的方法主要是针对神经网路分类模型的。对于非分类的模型,如何寻找类似近边界数据的特殊数据是能应用推断模型所有权方法的关键。


综上,本文提出的方法还有很大的探索空间。除此之外,未来的工作应该研究更多保护模型知识产权的新方法,防止模型盗窃者发起针对性的攻击,以更好的保护神经网络模型的知识产权。





