% !TeX root = ../main.tex
% -*- coding: utf-8 -*-
\chapter{总结与展望}\label{6}


\section{工作总结}

在本文中,我们讨论了以往研究中验证模型所有权的局限性,提出了用推断模型所有权代替验证。我们认为可以从数据驱动的角度抵御模型盗窃,即如果数据在源模型上存在一种可衡量的特性,那么这种特性也会被被盗模型所继承。因此,我们构建了一种有趣的近边界数据用以推断所有权,并设计了使用CW-$L_2$迭代添加小噪声的方法生成对抗性样本,这是初始的近边界数据。我们训练了一种基于DCGAN的近边界生数据成器用以将近边界数据私有化和扩展,实验测试了生成器能够显著地学习近边界数据的特征并生成新的数据。最后,我们设计了新的损失函数微调源模型的分类边界,得到了最终版本的源模型和近边界数据。我们在CIFAR-10,Heritage和Intel\_image数据集上进行评估,实验证明了我们的方法可以高置信度地推断模型所有权。

\section{工作展望}