% !TeX root = ../main.tex
% -*- coding: utf-8 -*-


\begin{zhaiyao}

深度神经网络(Deep Neural Network, DNN)因其优越的性能已经应用于科学研究和日常生活的方方面面,给人们的生活带来了极大地便利。一个高性能DNN模型离不开复杂的结构设计,覆盖全面的数据集和昂贵的计算资源以及漫长的训练时间,所以DNN模型具有很高的商业价值和知识产权价值。

近年来,模型盗窃行为时常出现,不法分子对DNN模型非法复制,派生和发布的行为都严重侵犯了模型所有者的知识产权。这引起人们逐渐重视模型的知识产权保护,许多研究者受到传统数字媒体水印的启发,从而设计模型水印和指纹用于验证模型所有权。然而,歧义性声明等攻击手段被用于破解模型水印和指纹,这对模型所有权验证工作造成了挑战。因此,本文提出了一种近边界数据作为获得模型所有权的证据,并创新性地提出了推断模型所有权而不是验证模型所有权。本文提出采用对抗性样本生成算法和生成对抗网络构造私有化的近边界数据,我们主要的观察结果是近边界数据在源模型和其衍生的盗窃模型中均表现出靠近分类边界的结果,模型所有者可以使用近边界数据作为DNN模型知识产权保护的技术支持。本文的主要工作如下:

\begin{enumerate}
	\renewcommand{\labelenumi}{\theenumi)}
	\item 揭示了当前DNN模型所有权验证方案的脆弱性并确认了数据驱动推断模型所有权的有效性。
	\item 提出了利用对抗性样本构造近边界数据以抵御模型窃取攻击。
	\item 设计了基于生成对抗网络的近边界数据生成器和提出了一种损失函数用以微调源模型的目标分类边界,增加推断模型所有权的置信度。
	\item 基于ResNet18和三个公开数据集进行了广泛的实验,实验结果证明了本文提出的近边界数据在推断模型所有权上的显著效果。
\end{enumerate}



\end{zhaiyao}




\begin{guanjianci}
知识产权保护;所有权推断;近边界数据;深度神经网络;生成对抗网络
\end{guanjianci}



\begin{abstract}


This is the abstract.

\end{abstract}



\begin{keywords}
Intellectual property protection; Ownership inference; Near-boundary data; Deep neural network; Generative adversarial network
\end{keywords} 