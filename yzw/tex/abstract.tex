% !TeX root = ../main.tex
% -*- coding: utf-8 -*-


\begin{zhaiyao}

深度神经网络(Deep Neural Network, DNN)训练代价昂贵是导致模型知识产权保护问题逐渐被重视的原因。近年来,模型盗窃行为时常出现,不法分子对DNN模型非法复制,派生和发布的行为都严重侵犯了模型所有者的知识产权。许多研究者受到传统数字媒体水印的启发,从而设计模型水印和指纹用于验证模型所有权。然而,歧义性声明等攻击手段被用于破解模型水印和指纹,这对模型所有权验证工作造成了挑战。因此,需要设计一种知识产权保护方法能够解决上述问题。本文通过相关工作的调研,分析目前知识产权保护方法存在的难点与挑战,提出了一种基于近边界数据的模型所有权推断方法,主要工作如下:

1)揭示了当前DNN模型所有权验证方案的脆弱性并确认了数据驱动推断模型所有权的有效性。模型水印和模型指纹的方法通常用是检测嵌入的水印或者通过特定触发集来验证所有权,这种方式在面对歧义攻击击等强攻击时,并不具备很强的鲁棒性。DNN模型通过数据集训练,所以不管是源模型还是其派生出来的模型,总会包含一定数据中的知识,本文确认了数据驱动推断模型所有权方法的有效性。

2)提出了利用对抗性样本构造近边界数据以抵御模型窃取攻击。对抗性样本一般位于模型分类边界上并且相较于其他不相关的模型,对抗性样本可以更好的转移到从原始模型派生出的模型上。因此,本文利用对抗性样本构造了近边界数据来推断模型所有权,抵御模型窃取攻击。

3)设计了基于DCGAN的近边界数据生成器和提出了一种损失函数用以微调源模型的目标分类边界,增加推断模型所有权的置信度。为了防止近边界数据被轻易复制,本文使用DCGAN的生成器生成我们私有的近边界数据。在此基础之上,重新设计了模型损失函数微调源模型,在保持DNN模型性能的情况下,以95\%以上的置信度成功推断模型所有权。

4)本文在三个公开数据集上对本文提出的方法做了详细的测试,实验结果证明了基于近边界数据推断模型所有权的有效性和鲁棒性


\end{zhaiyao}



\begin{guanjianci}
知识产权保护;所有权推断;近边界数据;深度神经网络;生成对抗网络
\end{guanjianci}



\begin{abstract}

Deep Neural Network(DNN) training is expensive, which is why the issue of model intellectual property protection has been increasingly emphasized. In recent years, model theft has been occurring frequently, with malicious actors illegally copying, deriving, and releasing DNN models, which seriously violates the intellectual property of the model owners. Many researchers have been inspired by traditional digital media watermarks and have designed model watermarks and fingerprints to verify model ownership. However, attack methods such as ambiguity declarations have been used to break model watermarks and fingerprints, posing a challenge to model ownership verification. Therefore, a method for protecting intellectual property needs to be designed to address these issues. This paper analyzes the difficulties and challenges of current methods for protecting intellectual property through related work, and proposes a model ownership inference method based on near-boundary data. The main contributions of this work are as follows:

1)We reveal the vulnerability of current DNN model ownership verification schemes and confirm the effectiveness of data-driven inference of model ownership. The methods of model watermark and model fingerprint are usually used to detect embedded watermarks or to verify ownership through specific trigger sets. However, these methods do not have strong robustness in the face of strong attacks such as ambiguity attacks. Since DNN models are trained on a dataset, both the source model and its derivatives will contain some knowledge from the data. This paper confirms the effectiveness of data-driven inference of model ownerships.
 
2)We propose the use of adversarial samples to construct near-boundary data to resist model theft attacks. Adversarial samples are generally located on the model classification boundary and can be better transferred to models derived from the original model compared to other unrelated models. Therefore, this paper uses adversarial samples to construct near-boundary data for ownership inference and to resist model theft attacks.

3)We design a near-boundary data generator based on DCGAN and propose a loss function to fine-tune the target classification boundary of the source model to increase the confidence of ownership inference. In order to prevent near-boundary data from being easily copied, this paper uses a DCGAN generator to generate private near-boundary data. On this basis, the model loss function is redesigned to fine-tune the source model, successfully inferring model ownership with over 95\% confidence while maintaining DNN model performance.

4)We conduct detailed tests on three public datasets to verify the effectiveness and robustness of the proposed method based on near-boundary data for model ownership inference. The experimental results demonstrate  the effectiveness and robustness of ownership inference based on near-boundary data.

\end{abstract}



\begin{keywords}
Intellectual property protection; Ownership inference; Near-boundary data; Deep neural network; Generative adversarial network
\end{keywords} 